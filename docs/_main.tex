% Options for packages loaded elsewhere
\PassOptionsToPackage{unicode}{hyperref}
\PassOptionsToPackage{hyphens}{url}
%
\documentclass[
]{book}
\usepackage{amsmath,amssymb}
\usepackage{lmodern}
\usepackage{iftex}
\ifPDFTeX
  \usepackage[T1]{fontenc}
  \usepackage[utf8]{inputenc}
  \usepackage{textcomp} % provide euro and other symbols
\else % if luatex or xetex
  \usepackage{unicode-math}
  \defaultfontfeatures{Scale=MatchLowercase}
  \defaultfontfeatures[\rmfamily]{Ligatures=TeX,Scale=1}
\fi
% Use upquote if available, for straight quotes in verbatim environments
\IfFileExists{upquote.sty}{\usepackage{upquote}}{}
\IfFileExists{microtype.sty}{% use microtype if available
  \usepackage[]{microtype}
  \UseMicrotypeSet[protrusion]{basicmath} % disable protrusion for tt fonts
}{}
\makeatletter
\@ifundefined{KOMAClassName}{% if non-KOMA class
  \IfFileExists{parskip.sty}{%
    \usepackage{parskip}
  }{% else
    \setlength{\parindent}{0pt}
    \setlength{\parskip}{6pt plus 2pt minus 1pt}}
}{% if KOMA class
  \KOMAoptions{parskip=half}}
\makeatother
\usepackage{xcolor}
\usepackage{color}
\usepackage{fancyvrb}
\newcommand{\VerbBar}{|}
\newcommand{\VERB}{\Verb[commandchars=\\\{\}]}
\DefineVerbatimEnvironment{Highlighting}{Verbatim}{commandchars=\\\{\}}
% Add ',fontsize=\small' for more characters per line
\usepackage{framed}
\definecolor{shadecolor}{RGB}{248,248,248}
\newenvironment{Shaded}{\begin{snugshade}}{\end{snugshade}}
\newcommand{\AlertTok}[1]{\textcolor[rgb]{0.94,0.16,0.16}{#1}}
\newcommand{\AnnotationTok}[1]{\textcolor[rgb]{0.56,0.35,0.01}{\textbf{\textit{#1}}}}
\newcommand{\AttributeTok}[1]{\textcolor[rgb]{0.77,0.63,0.00}{#1}}
\newcommand{\BaseNTok}[1]{\textcolor[rgb]{0.00,0.00,0.81}{#1}}
\newcommand{\BuiltInTok}[1]{#1}
\newcommand{\CharTok}[1]{\textcolor[rgb]{0.31,0.60,0.02}{#1}}
\newcommand{\CommentTok}[1]{\textcolor[rgb]{0.56,0.35,0.01}{\textit{#1}}}
\newcommand{\CommentVarTok}[1]{\textcolor[rgb]{0.56,0.35,0.01}{\textbf{\textit{#1}}}}
\newcommand{\ConstantTok}[1]{\textcolor[rgb]{0.00,0.00,0.00}{#1}}
\newcommand{\ControlFlowTok}[1]{\textcolor[rgb]{0.13,0.29,0.53}{\textbf{#1}}}
\newcommand{\DataTypeTok}[1]{\textcolor[rgb]{0.13,0.29,0.53}{#1}}
\newcommand{\DecValTok}[1]{\textcolor[rgb]{0.00,0.00,0.81}{#1}}
\newcommand{\DocumentationTok}[1]{\textcolor[rgb]{0.56,0.35,0.01}{\textbf{\textit{#1}}}}
\newcommand{\ErrorTok}[1]{\textcolor[rgb]{0.64,0.00,0.00}{\textbf{#1}}}
\newcommand{\ExtensionTok}[1]{#1}
\newcommand{\FloatTok}[1]{\textcolor[rgb]{0.00,0.00,0.81}{#1}}
\newcommand{\FunctionTok}[1]{\textcolor[rgb]{0.00,0.00,0.00}{#1}}
\newcommand{\ImportTok}[1]{#1}
\newcommand{\InformationTok}[1]{\textcolor[rgb]{0.56,0.35,0.01}{\textbf{\textit{#1}}}}
\newcommand{\KeywordTok}[1]{\textcolor[rgb]{0.13,0.29,0.53}{\textbf{#1}}}
\newcommand{\NormalTok}[1]{#1}
\newcommand{\OperatorTok}[1]{\textcolor[rgb]{0.81,0.36,0.00}{\textbf{#1}}}
\newcommand{\OtherTok}[1]{\textcolor[rgb]{0.56,0.35,0.01}{#1}}
\newcommand{\PreprocessorTok}[1]{\textcolor[rgb]{0.56,0.35,0.01}{\textit{#1}}}
\newcommand{\RegionMarkerTok}[1]{#1}
\newcommand{\SpecialCharTok}[1]{\textcolor[rgb]{0.00,0.00,0.00}{#1}}
\newcommand{\SpecialStringTok}[1]{\textcolor[rgb]{0.31,0.60,0.02}{#1}}
\newcommand{\StringTok}[1]{\textcolor[rgb]{0.31,0.60,0.02}{#1}}
\newcommand{\VariableTok}[1]{\textcolor[rgb]{0.00,0.00,0.00}{#1}}
\newcommand{\VerbatimStringTok}[1]{\textcolor[rgb]{0.31,0.60,0.02}{#1}}
\newcommand{\WarningTok}[1]{\textcolor[rgb]{0.56,0.35,0.01}{\textbf{\textit{#1}}}}
\usepackage{longtable,booktabs,array}
\usepackage{calc} % for calculating minipage widths
% Correct order of tables after \paragraph or \subparagraph
\usepackage{etoolbox}
\makeatletter
\patchcmd\longtable{\par}{\if@noskipsec\mbox{}\fi\par}{}{}
\makeatother
% Allow footnotes in longtable head/foot
\IfFileExists{footnotehyper.sty}{\usepackage{footnotehyper}}{\usepackage{footnote}}
\makesavenoteenv{longtable}
\usepackage{graphicx}
\makeatletter
\def\maxwidth{\ifdim\Gin@nat@width>\linewidth\linewidth\else\Gin@nat@width\fi}
\def\maxheight{\ifdim\Gin@nat@height>\textheight\textheight\else\Gin@nat@height\fi}
\makeatother
% Scale images if necessary, so that they will not overflow the page
% margins by default, and it is still possible to overwrite the defaults
% using explicit options in \includegraphics[width, height, ...]{}
\setkeys{Gin}{width=\maxwidth,height=\maxheight,keepaspectratio}
% Set default figure placement to htbp
\makeatletter
\def\fps@figure{htbp}
\makeatother
\setlength{\emergencystretch}{3em} % prevent overfull lines
\providecommand{\tightlist}{%
  \setlength{\itemsep}{0pt}\setlength{\parskip}{0pt}}
\setcounter{secnumdepth}{5}
\usepackage{booktabs}
\ifLuaTeX
  \usepackage{selnolig}  % disable illegal ligatures
\fi
\usepackage[]{natbib}
\bibliographystyle{plainnat}
\IfFileExists{bookmark.sty}{\usepackage{bookmark}}{\usepackage{hyperref}}
\IfFileExists{xurl.sty}{\usepackage{xurl}}{} % add URL line breaks if available
\urlstyle{same} % disable monospaced font for URLs
\hypersetup{
  pdftitle={モダンなアラビア語学習法: 私がどのようにアラビア語を(ある程度)習得したか},
  pdfauthor={持田 慧},
  hidelinks,
  pdfcreator={LaTeX via pandoc}}

\title{モダンなアラビア語学習法: 私がどのようにアラビア語を(ある程度)習得したか}
\author{持田 慧}
\date{2022-12-13}

\begin{document}
\maketitle

{
\setcounter{tocdepth}{1}
\tableofcontents
}
\hypertarget{index}{%
\chapter*{はじめに}\label{index}}
\addcontentsline{toc}{chapter}{はじめに}

بسم الله الرحمن الرحيم

はじめまして、持田です\footnote{「お前さん、何者だ」と思われた方は\protect\hyperlink{author}{自己紹介}へ}。はじめましてじゃない人は、「あぁ、こいつか」とでも思っていただければよいかと。この文書は、私がカタール派遣に向けて何をしたか、また派遣中にどのように勉強していたかを記したものです。また、具体的な参考文献や Web サイトについても紹介するので、「学習法なんぞどうでもいい」という人も、そこだけは参考になると思います。

目次は左のリストからどうぞ

\hypertarget{get}{%
\section*{これを読んで得られるもの}\label{get}}
\addcontentsline{toc}{section}{これを読んで得られるもの}

\begin{itemize}
\tightlist
\item
  アラビア語の勉強の仕方\footnote{これは他の言語にも適用可能。アラビア語の時よりはむしろ、インドネシア語学習の経験から得たものと言えるかもしれない。}
\item
  \texttt{Anki}\footnote{分散学習を利用した暗記補助ソフトウェア}の使い方
\item
  文献の情報
\item
  アラビア語の豆知識
\item
  標準語と口語の違い
\end{itemize}

\hypertarget{background}{%
\section*{これを書くに至った背景}\label{background}}
\addcontentsline{toc}{section}{これを書くに至った背景}

何にしてもその背景を知ることは重要です。つまり、\textbf{この項目は飛ばさずに読んでいただきたい}訳です。なんで、持田が色々とほったらかしてでも、こんなに長ったらしい文書を書いているかが分かります。

\hypertarget{level}{%
\subsection*{カタール語学学校のレベルの話}\label{level}}
\addcontentsline{toc}{subsection}{カタール語学学校のレベルの話}

カタールに到着したら、まず素養試験が行われ、その試験の結果に基づいてクラスの配分が決定されます。つまり、そこで良い点数を取ればより上のクラスへ、低ければ下のクラスに分けられるということです。 では、具体的にどのようなクラスがあるのかと言うと、私の知る限りでは、

\begin{itemize}
\tightlist
\item
  Level 1 入門アラビア語課程(Beginner Arabic Course)
\item
  Level 2 初級アラビア語課程(Elementary Arabic Course)
\end{itemize}

これらの 2 つがあります。私の場合は、レベル 2 の初級コースに入ることができました。 一方で、一緒に来た相方は、どちらもレベル 1 の入門コースに区分されていた。

まず、入門コースの説明をすると、何もアラビア語を知らない者を対象にしたもので、文字の読み方から始まる。 そのため、相方は\textbf{「かなりつまらない」}述べていた。

次に、自分の初級コースは、名詞文から始まり、すぐに動詞文などに入っていきました。長文に関しても勉強しました。文法事項としては、あまり難しいものではありませんでしたが、\textbf{かなり語彙力は伸びたし、会話力も上がった}と思います。

\hypertarget{nut_shell}{%
\subsection*{要は?}\label{nut_shell}}
\addcontentsline{toc}{subsection}{要は?}

\textbf{つまり、何が言いたいかというと、入門クラスに入るのはもったいないということ!!} \textbf{しっかり勉強して、初級クラスに入るとより得られるものが多い!!}

そして、ハッキリと言うが、授業だけでは知識が不足するので、自分でやる必要があるということ。

\hypertarget{author}{%
\chapter*{自己紹介}\label{author}}
\addcontentsline{toc}{chapter}{自己紹介}

今更ではありますが、「お前は誰だ」と思われていることでしょうから、簡単に自分が何者であるかについて述べていきます。 興味のない人は読み飛ばし推奨。

名前: 持田 慧\\
出身地: 東京都西多摩郡瑞穂町\footnote{摩天楼なんてものは存在しない田舎です。}\\
趣味: 言語学習\footnote{英語は言うまでもない。あと、インドネシア語も多少できる。これを書いている段階ではトルコ語にも手を出している。}

令和3年度(2021年)に、カタール派遣された人。コロナで前年度の派遣が中止されたため、2年振りの派遣となりました。この年は、ちょうど米軍のアフガニスタン撤退でごたついた時期でしたので、色々とカブール撤退の裏話を聞けたりと面白い派遣になりました。

\hypertarget{principle}{%
\chapter{基本方針}\label{principle}}

一番最初に声を大きくして言いたいことは、

{\textbf{言語は音で構成されている}}

これは、私の学習方法の\textbf{根幹}をなしているものであり、単語をやるにしても、長文をやるにしても、会話の練習をするにしても、\textbf{全ての事項に当てはまる原則}である。

そして、その原則から私は可能な限り、\textbf{コンピューターを使用}した学習を行っている。

コンピューターの利点

\begin{itemize}
\tightlist
\item
  音声を簡単に\textbf{再生}できる
\item
  情報を簡単に\textbf{参照}できる
\item
  情報を簡単に\textbf{追加}できる
\item
  面倒な管理を\textbf{自動}でやってくれる
\end{itemize}

もちろん、紙などのアナログな媒体を使うこともある。\\
長文をノートに写して、そこにメモを書いていくというやり方で、タブレット端末等がなければ、この方法が良いだろう。

\hypertarget{part-ux7406ux8ad6ux7de8-wip}{%
\part{理論編 {[}WIP{]}}\label{part-ux7406ux8ad6ux7de8-wip}}

\hypertarget{theory_arabic}{%
\chapter{アラビア語}\label{theory_arabic}}

\hypertarget{colloquial}{%
\section{口語(アーンミーヤ,العامية)について}\label{colloquial}}

ご存知の通り(ご存知でない方は新たなことを知る機会です)、アラビア語は、

\begin{itemize}
\tightlist
\item
  標準語 フスハー(Ar: الفصحى, En: Modern Standard Arabic, MSA)
\item
  口語 アーンミーヤ(En: Spoken Arabic, Colloquial Arabic)
\end{itemize}

の2つの種類がある。

\hypertarget{distr_learn}{%
\chapter{分散学習}\label{distr_learn}}

\hypertarget{part-ux5b9fux8df5ux7de8-wip}{%
\part{実践編 {[}WIP{]}}\label{part-ux5b9fux8df5ux7de8-wip}}

\hypertarget{resource}{%
\chapter{学習素材情報}\label{resource}}

ここに参考となる学習素材を紹介し、それに関するコメントを書いていく。 タイトルをリンクにしているため、クリックすれば出版社(一部はECサイト)のWebサイトに飛ぶ。

\begin{enumerate}
\def\labelenumi{\arabic{enumi}.}
\tightlist
\item
  【文法\&問題】\href{https://www.hakusuisha.co.jp/book/b348647.html}{竹田敏之『アラビア語表現とことこトレーニング』白水社、2018年。} 特徴は文法解説と問題が見開きで構成されていて、解説を見ながら問題に取り組み事ができる。つまり楽。
\item
  【単語集】\href{(}{鷲見朗子編『例文で学ぶアラビア語単語集』大修館書店、2019年。}\url{https://www.taishukan.co.jp/book/b471933.html}) 分野ごと(例えば、学校やレストランなどの場所や経済・ITなどの概念など)にイラストを交えて単語が載せられているため、調べ物をするときに便利。
\item
  【文法\&会話集】\href{https://www.bookdepository.com/Complete-Spoken-Arabic-Arabian-Gulf-Beginner-Intermediate-Course-Frances-Smart/9781444105469?ref=grid-view\&qid=1669905445206\&sr=1-1}{Smart, Jack and Frances Altorfer, \emph{Complete Spoken Arabic of the Gulf}, Teach Yourself, 2010.} 湾岸方言のテキスト。
\item
  【会話集\&単語集】Fazza, Hany, \emph{Qatari Phrasebook}, V. 1.0.4, Georgetown University in Qatar, Android and iOS, 2015. 湾岸・カタール方言のフレーズを集めたアプリケーション。ジョージタウン大学カタール分校が開発したもので、無料で利用できる。AndoroidとiOSどちらでも利用可能。音声の再生も可能でかなり利便性が高い。
\item
  a
\end{enumerate}

\hypertarget{appendix_a}{%
\chapter*{付録A: よく使うアラビア語湾岸方言の単語・フレーズ}\label{appendix_a}}
\addcontentsline{toc}{chapter}{付録A: よく使うアラビア語湾岸方言の単語・フレーズ}

湾岸方言と上に書いているが、標準語も混じっているので、あしからず。

思いつきで書いているので、リストの順番に特に意味はない。

参考にしたもの:

\begin{itemize}
\tightlist
\item
  経験
\item
  Qatari Phrasebook
\item
  Conversational Arabic Quick and Easy (Qatari Dialect)
\item
  Complete Spoken Arabic
\end{itemize}

\begin{longtable}[]{@{}
  >{\raggedleft\arraybackslash}p{(\columnwidth - 6\tabcolsep) * \real{0.1210}}
  >{\raggedright\arraybackslash}p{(\columnwidth - 6\tabcolsep) * \real{0.1032}}
  >{\raggedright\arraybackslash}p{(\columnwidth - 6\tabcolsep) * \real{0.1281}}
  >{\raggedright\arraybackslash}p{(\columnwidth - 6\tabcolsep) * \real{0.6406}}@{}}
\toprule()
\begin{minipage}[b]{\linewidth}\raggedleft
アラビア語
\end{minipage} & \begin{minipage}[b]{\linewidth}\raggedright
発音
\end{minipage} & \begin{minipage}[b]{\linewidth}\raggedright
日本語
\end{minipage} & \begin{minipage}[b]{\linewidth}\raggedright
備考(と言う名の思いつくままに書いたコメント)
\end{minipage} \\
\midrule()
\endhead
تَمام & tamām & Ok, 了解 & 恐らく、持田がカタール派遣間で一番使った単語。ちなみにトルコ語でも同じ意味だからトルコ人にも通じる。 \\
شَخْبارَك (m.)شـَخْبارَچ (f.) & shakhbārakshakhbārach & お元気ですか? & كيف حارك の代わりに使える。これを使うとカタール人に驚かれる。ちなみに女性形を使ったことは一度もなかった。あと、چは本来のアラビア語にはない音でchと発音する。ペルシャ語の影響かな? \\
\ldots{} أبي & 'abbī \ldots{} & 私は\ldots が欲しい(したい) & お茶が欲しければأبي چاي、行きたい所があれば、 أبي أروحと言った感じでかなり万能。動詞も未完了形を付けれる。お父さんではない。 \\
سِفَارة & sifārah & 大使館 & こんなの使うわけないだろ~、と思うなかれ。意外と重要な単語。大使館に呼ばれるかもしれないからね(実際、派遣初日に呼ばれて焦った)。 \\
جَواز السَفَر & jawāz ssafar & パスポート & ビザの更新が必要な時があるので覚えておこう。 \\
عَسْكَرعسكري(adj) & 3skar3skarī & 軍隊 & 一般人はいらないが立場上必要になる単語。形容詞形では、軍人という意味もあるから自己紹介にも使える。مُعَسْكَرで「駐屯地, 基地」 \\
مُلْحَق العسكري & mulHaq l-3skarī & 駐在武官 & クウェートの防衛駐在官(カタールと兼轄)がカタールに来ることもあるから、覚えといて損はない。 \\
مُرَشَح الضَابِط & murashaH DDābiT & 士官候補生 & 単にمرشحでも通じる。 \\
أنا مِسْتَعِد & anā mista3id & 準備okです & よし行くぞー、というときに使おう \\
مُشكِلة & mushkilah & 問題 & 問題は問題でもproblemの方。何か困ったときやダメなときは、これを使って助けを求めよう。それにしても発音が「虫けら」みたいで面白い。 \\
في \ldots؟ & fī \ldots{} ? & ~はありますか? & 湾岸方言の言い方。かなり単純化されててすごい楽。 \\
مافي & māfī & (ものなどが)ない & これもかなり使う。 مافي مشكلة 「問題ない」など \\
تَكْمِيل & takmīl & 点呼 & このつづりが正しいか検証できてないので、正確なものかは分からない。点呼はあちらでもあるので注意! \\
ساعة كم الحين؟ & sā3ah kam alHīn & 今何時ですか? & 標準語だと疑問詞が最初に来るはずだが、湾岸方言だと逆となる。 \\
الحِين & alHīn & 今 & 標準語はالأن \\
فْلوس & flūs & お金 & お金は大事。!!مافي فلوس \\
مَشْكور & mashkūr & ありがとう & شكراでいいかなと思いつつ、一応追加しておく。 \\
رُوح/ يِروح & rūH/ yirūH & 行く & 行きたい場所を伝えよう! أروح سيتي سنتر(シティーセンター{[}モールの名前{]}に行きます。) \\
وَين & wain & どこ(疑問詞) & 発音はどちらかというweinな気がする。أينの代わりに使える。 \\
مَاي & māi & 水 & 標準語のماءより発音が楽。カタールの水はとても安い(どうでもいい情報)。 \\
ُما فَهَمت & mā fahamtu & わかりません & 自分が理解していないということを伝えるのって勇気いるよね。 \\
ماعرف & mā3araf & 知りません & 知らないことはどんどん聞こう! \\
أنا جَوْعان & 'anā jaw3ān & お腹空きました & カタール人は店で大量にモノを頼む。全部食べる必要はない(食べきるの無理)。 \\
سَلَّة قَمامة & sallah qamāmah & ゴミ箱 & カタールに分別なんて概念は存在しない。 \\
خَلاص & khalāS & 終わりました & 終わりのないものなどない。つまり必須の単語。 \\
مُتَقَّدِم & mutaqqadim & 3年生 & 自分の学年は言える様にしておこう。!!يا متقدم \\
خَريج & kharīj & 4年生 & 4年生らしいが辞書だと「卒業生」という意味。多分通じるはず\ldots{} \\
مُسْتَجِد & mustajid & 1年生 & 原義は「新人」 \\
مُتَوَسِّط & mutawassiT & 2年生 & 原義は「中間, 平均」 \\
كيس & kīs & 袋 & ゴミ袋とかレジ袋とか、そういった類のもの。欲しい時あるでしょう? \\
خَليج & Khalij & 湾岸 & 日本人はペルシャ湾というが、カタールでは「アラビア湾, خليج العربي」という。まぁ、そう呼ぶよなぁ。 \\
\ldots{} مُمْكِن & mumkin \ldots{} & \ldots{} できますか? & \ldots の部分には未完了動詞や名詞が入る。 \\
أَخْتاج & 'akhtāj & 私は~が必要です & 「欲しい」よりも強い感じだよね。 \\
شو اسمك؟شو اسمچ؟ & shi-smakshi-smach & あなたの名前は何ですか? & شوの発音が文字通りではないので注意。شِنو「何」を省略したものと思われる。女性形は使っ(ry \\
مُكِيف & mukīf & エアコン & カタールの空調は強烈に寒いので半袖短パンはやめといた方がいい。寒い時は、سَكَّرْ مكيف「空調を切って」と言おう。 \\
دَقِيقة & daqīqah / dagīgah & ちょっと待ってください & 原義は「分」。湾岸方言だとقはgの音になる。まぁ、そっちの方が楽だよね。 \\
أَكِيد & akīd & もちろん & of couseのような使い方ができる。 \\
إشلون تِقول هاي بالعربي؟ & shlūn tigūl hāy bil-3arabī & これはアラビア語で何と言いますか? & هذا = هايكيف = اشلون

だから標準語だとكيف تَقول هاذا بالعربية؟となる(はず) \\
شَنْطة & shanTah & かばん & 同じ意味のحقيبةより聞いた単語。トルコ語もçantaだから、ペルシャ語が語源じゃないかな。 \\
\bottomrule()
\end{longtable}

\hypertarget{appendix_b}{%
\chapter*{付録B: アラブ音楽の世界 {[}WIP{]}}\label{appendix_b}}
\addcontentsline{toc}{chapter}{付録B: アラブ音楽の世界 {[}WIP{]}}

当付録は、個人的に好きなアラブ音楽の世界へ招待しようという試みの下で書かれたものである。\\
かなり、個人の趣味\footnote{ちなみに持田の大好物は、伝統音楽と電子音楽の融合である。}が反映されているため、「こんなのあるんだなー」程度に思っていただければ幸いである。

これを読んで興味を持ったならば、音楽評論家の\href{https://www.chez-salam.com}{サラーム海上}氏のWebサイトへ訪問すると、良い発見があるだろう。彼のJ-WAVEラジオ番組「\href{https://www.j-wave.co.jp/original/dc3/}{Oriental Music Show}」を聞けば、たちまちワールドミュージックの世界に引き込まれるだろう。

\hypertarget{a-wa}{%
\section*{A-WA}\label{a-wa}}
\addcontentsline{toc}{section}{A-WA}

A-WAは、イエメン系イスラエル人姉妹のトリオで、「Habib Galbi」は上で紹介した「Oriental Music Show」のオープニング曲となっている。

\hypertarget{zenobia-ux632ux646ux648ux628ux64aux627}{%
\section*{Zenobia زنوبيا}\label{zenobia-ux632ux646ux648ux628ux64aux627}}
\addcontentsline{toc}{section}{Zenobia زنوبيا}

Zenobiaはパレスチナのダンス・ミュージック・デュオだ。
彼らは、レヴァント地方の伝統舞踊音楽である「ダブケ」をシンセサイザーを用いたテクノ音楽に仕立て上げている。上の曲「Halak Halak」では、シンセで中東の伝統楽器「ミジュウィズ」の音を再現している。あまりにも好きすぎるので、\textbf{CDを買った}。

\hypertarget{rahim-alhaj-ux631ux62dux64aux645-ux627ux644ux62dux627ux62c}{%
\section*{Rahim AlHaj رحيم الحاج}\label{rahim-alhaj-ux631ux62dux64aux645-ux627ux644ux62dux627ux62c}}
\addcontentsline{toc}{section}{Rahim AlHaj رحيم الحاج}

\hypertarget{naseer-shamma-ux646ux635ux64aux631-ux634ux645ux629}{%
\section*{Naseer Shamma نصير شمة}\label{naseer-shamma-ux646ux635ux64aux631-ux634ux645ux629}}
\addcontentsline{toc}{section}{Naseer Shamma نصير شمة}

\hypertarget{appendix_c}{%
\chapter*{付録C: どのようにこれを書いたか}\label{appendix_c}}
\addcontentsline{toc}{chapter}{付録C: どのようにこれを書いたか}

このサイトは、\texttt{RStudio}と\texttt{bookdown}パッケージを使用して書いた。

\hypertarget{env}{%
\section*{執筆環境}\label{env}}
\addcontentsline{toc}{section}{執筆環境}

\begin{Shaded}
\begin{Highlighting}[]
\FunctionTok{sessionInfo}\NormalTok{()}
\end{Highlighting}
\end{Shaded}

\begin{verbatim}
## R version 4.2.2 (2022-10-31 ucrt)
## Platform: x86_64-w64-mingw32/x64 (64-bit)
## Running under: Windows 10 x64 (build 22621)
## 
## Matrix products: default
## 
## locale:
## [1] LC_COLLATE=Japanese_Japan.utf8  LC_CTYPE=Japanese_Japan.utf8   
## [3] LC_MONETARY=Japanese_Japan.utf8 LC_NUMERIC=C                   
## [5] LC_TIME=Japanese_Japan.utf8    
## 
## attached base packages:
## [1] stats     graphics  grDevices utils     datasets  methods   base     
## 
## loaded via a namespace (and not attached):
##  [1] compiler_4.2.2  magrittr_2.0.3  fastmap_1.1.0   bookdown_0.30  
##  [5] cli_3.4.1       htmltools_0.5.3 tools_4.2.2     rstudioapi_0.14
##  [9] yaml_2.3.6      stringi_1.7.8   rmarkdown_2.17  knitr_1.41     
## [13] stringr_1.4.1   digest_0.6.30   xfun_0.35       rlang_1.0.6    
## [17] evaluate_0.18
\end{verbatim}

\hypertarget{ref}{%
\section*{参考文献}\label{ref}}
\addcontentsline{toc}{section}{参考文献}

\begin{itemize}
\tightlist
\item
  Desirée De Leon and Alison Hill, \href{https://rstudio4edu.github.io/rstudio4edu-book/}{\emph{rstudio4edu}}, 2019.
\item
  Yihui Xie, Christophe Dervieux and Emily Riederer, \href{https://bookdown.org/yihui/rmarkdown-cookbook/}{\emph{R Markdown Cookbook}}, 2022. (日本語訳:『\href{https://gedevan-aleksizde.github.io/rmarkdown-cookbook/}{R Markdown クックブック}』)
\item
  Yihui Xie, \href{https://bookdown.org/yihui/bookdown/}{\emph{bookdown: Authoring Books and Technical Documents with R Markdown}}, 2022.
\item
  宋財泫、矢内勇生『\href{https://www.jaysong.net/RBook/}{私たちのR: ベストプラクティスの探究}』2022 年。
\item
  松村優哉他、『改定2版 Rユーザーのための RStudio{[}実践{]}入門: tidyverseによるモダンな分析フローの世界』技術評論社、2021 年。
\end{itemize}

  \bibliography{references.bib}

\end{document}
